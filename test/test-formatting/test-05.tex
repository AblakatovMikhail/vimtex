Therefore, in that case the observed dynamics cannot be truly
Markovian. We can still approximately capture the observed density dynamics
with eq.~\ref{eq:fpe}; however the local coefficients are then effectively
averages over the co-evolving unobserved distribution. For example, $\vec
v(\vec x,t)=\int p(\vec x^u|\vec x,t)v(\vec x,\vec x^u,t)dx^u$. This has some
important consequences. First, the effective Markovian coefficients $\vec v,
\lambda, D$, will generally depend on time, reflecting the time evolution of
the hidden $p(\vec x^u)$ over the time scale of the experiment. Second, the
effective coefficients may also depend on the initial condition: a cell
population started concentrated around label $\vec x_0$ implies a certain
distribution over $x^u$ when sampled at label $\vec x$ some time later; another
population started around $\vec x_1$ carries a different history when it, too,
visits $\vec x$ later. Therefore, for instance the effective drift $\vec v(\vec
x)$ will be different in the two cases. We conclude that, if unobserved degrees
of freedom have dynamics on the time scale of the experiment eq.~\ref{eq:fpe}
is an approximation whose coefficients become \emph{non-universal} (dependent
on intial conditions) and \emph{time-dependent}.
